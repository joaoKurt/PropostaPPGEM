\section{Introdução}

% 1. Introdução
%    1. Motivação
%       1. entre as duas conectar c como e pq simular
%    2. Lacunas (Revisão Bibliográfica, estado da arte)
%    3. Justificativa
%    4. Objetivos (e Hipótese)
%    5. Contribuição Científica

% Importância de trabalhar com combustão no contexto global.

% Como se simula combustão

% Necessidade de modelos HMT para partículas em simulações de combustão e sua importância

% Explain need for multicomponent modeling in current global context
% Explain need for modeling the interior of the droplet
% Explain need for droplet combustion model

% GAP: lack of models for single droplet combustion, modeling way behind pure evaporation
% GAP: lack of consistent multicomponent version of evaporation models and lack of multicomponent single-droplet burning models.
% GAP: lack of computationally affordable methods that account for droplet interior AND advanced evaporation/SDC models.

A demanda pela transição energética e pela descarbonização da economia busca alternativas pela substituição dos combustíveis fósseis nos setores de energia, transporte, indústria. 
Alguns setores, como o energético e o de transportes urbano de baixa carga, têm mostrado grande progresso no uso de energias renováveis e na eletrificação, respectivamente \source{}.
Porém, combustíveis fósseis são extremamente difíceis de substituir em outros setores, como o de transporte de carga pesada, aéreo e marítimo, e em algumas indústrias como aço e cimento \source{}.
Isso se deve, em parte, à elevada densidade energética e energia específicas dos combustíveis fósseis, especialmente os combustíveis líquidos \source{}, mas também à extensa infraestrutura de transporte, distribuição e armazenamento para combustíveis líquidos já existente. \source{}
Além disso, uma análise histórica indica que a transição para fontes renováveis se dará ao longo de décadas \cite{MasriA2021}.

Nota-se que processos de combustão continuarão relevates nas próximas décadas e soluções devem ser procuradas para concicliar essa demanda com os esforços de transição energética e descarbonização da economia.
A comunidade científica busca, então, três caminhos: (i) novos combustíveis; (ii) novas origens para os mesmos combustíveis; (iii) melhorar a eficiência dos motores a combustão e reduzir a formação de poluentes.
A primeria abordagem consiste em desevolver o uso de combustíveis que não poluam -- como hidrogênio, amônia e pós metálicos -- ou que poluam menos -- como OMEs e naphta (ver \cite{MasriA2021}) -- ou que venham de outras fontes -- como o etanol e o metanol.
Dentre as fontes não fósseis, se destacam os bio-combustíveis, produzidos a partir de biomassa renovável, e o eletrocombustíveis, sintetizados a partir de água, gás carbônico e energia de fontes renováveis.
Em particular o etanol é importante para o Brasil ... \yellow{TODO}
% São exemplos da primeira abordagem o interesse crescente na combustão da amônia \source{}, do hidrogênio \source{}, do metanol \source{} e do etanol \source{} -- conhecidos Brasil --, de pós metálicos \source{} e outros \source{}. 
% Pela segunda abordagem, se destacam os eletrocombustíveis (\emph{e-fuels}), sintetizados a partir de gás carbônico, água e energia elétrica, e os bio-combustíveis

Independente da origem do combustível, o processo de combustão deve ser compreendido para que motores e queimadores eficientes e com baixa formação de poluentes sejam desenvolvidos.
Para tanto é necessário pesquisa em combustão, que pode ser estruturada em trabalhos experimentais ou trabalhos de modelagem.
Em particular, a modelagem da combustão turbulenta de sprays é bem desafiadora, devido aos diversos fênomenos envolvidos.
São escoamentos particulados, dispersos e reativos, que envolvem fenômenos multi-escala, multifásicos e, frequentemente, combustíveis multicomponente.
Revisando a literatura mais recente, o autor notou a ausência de estudos investigando a influência de modelar a combustão de gota isolada na estrutura da chama.
Ademais, há um atraso em termos de complexidade na utilização de modelos de transferência de massa e calor (HMT - \emph{heat and mass transder}) em gotas nas simulações de larga escala, em comparação com simulações de uma única gota.
% Revisando a literatura mais recente, constatou-se uma dificuldade de identificar e modelar diferentes modos de combustão de spray.
% \colorbox{cyan}{Em particular, como a combustão isolada de gotículas afeta a estrutura da chama.}

Na área da modelagem, que é foco dessa pesquisa
Como se entende o processo ... experimento e modelagem ... foco em modelagem e simulação ... CFD.
A simulação de combustão gasosa já é bem complexa, em parte devido à coexistência de fenômenos químicos e fenômenos de transporte, assim pela influência da turbulência e as interações entre esses aspectos.
A combustão de sprays traz a complexidade adicional de possuir duas fases: uma fase contínua e gasosa, e uma fase dispersa, geralmente líquida.
Escoamentos bifásicos como no caso de aerosóis ou sprays já são bem complexos, mesmo quando não reativos.
% Talvez por isso, para a simulação de escoamentos reativos de sprays, para parte química, utiliza-se técnicas desenvolvidas para a combustão gasosa.
% Combustíveis ... a maioria é líquido ... maior complexidade e desafios ...
% Foco em combustão de sprays líquidos ... 

A modelagem para a sinulação de escoamentos reativos, dispersos e multifásicos, como na combustão de sprays, requer três partes: (i) a modelagem da fase contínua gasosa; (ii) a modelagem da fase dispersa, líquida; (iii) a modelagem da química homogênea, i.e. da combustão na fase gasosa.
Para a parte (iii), a maioria dos trabalhos de combustão de sprays utiliza metodologias desenvolvidas para a combustão gasosa, considerando as gotas apenas como fontes de vapor de combustível.
Isso corresponde 

% Necessidade e dificuldade de simular a combustão turbulenta de Sprays... Mencionar revisões ... que tipos de simulação tem .. dificuldade e necessidade de simulações de larga escala ... normalmente com química tabelada.

% Âmbito e objetivo desse trabalho.

% Necessidade e dificuldade de representação multicomponente da gota.

% Necessidade e dificuldade de representar o interior da gota.

% Necessidade e dificuldade de representar diferentes modos de combustão de gota -> combustão de gota isolada.

% No cenario atual de busca pela transição energética e descarbonização, a pesquisa em combustão é mais importânte do que nunca.
% Nesse contexto, os principais objetivos são reduzir emissões, adaptar os modelos para novos combustíveis mais limpos e para novos modos de combustão, necessários para motores mais limpos.

% Análises históricas indicam que a transição para fontes de energia renováveis se dará ao longo de décadas \cite{MasriA2021}, [MASRI 1, 3 também]. Até lá, combustíveis fosseis mantém o seu domínio.
% Além disso, no setor de transportes, a eletrificação dificilmente substituirá combustíveis fósseis em veículos de cargas pesadas ou para transporte aéreo ou marítmo, mesmo considerando previsões otimistas para o desevolvimento de baterias \cite{MasriA2021}. O mesmo pode ser dito para processos industriais intensivos em energia, como cimento e aço.

% Uma opçao para a descarbonização desses setores é o uso de eletrocombustíveis, também chamados de \emph{e-fuels}, \emph{power-to-x} (PtX) ou \emph{powerfuels}.
% Estes são combustíveis líquidos ou gasosos produzidos a partir de água, gas carbônico ($\text{CO}_2$), eventualmente nitrogênio $\text N_2$ e energias de fontes renováveis. 
% Pode-se produzir, por exemplo: hidrogênio, metano, metanol, hidrocarbonetos líquidos (processo Fischer-Tropf), éteres de oximetileno (OMEs) e amônia (processo Haber-Bosch). source{MASRI 9}.
% Os combustíveis PtX podem utilizar a infraestrutura de transporte, distribuição e armazenamento já existente dos combustíveis fósseis \source{MASRI 9-13} e podem chegar a ser neutros em termos de emissões, apesar que ainda mais caros que combustíveis de origem fóssil \source{MASRI 42, 43}. 

% Outra alternativa para combustíveis de origem fóssil são os combustíveis verdes ou bio-combustíveis, produzidos a partir de platas. 
% Por exemplo, o etanol produzidos a partir do bagaço de cana, tecnologia desenvolvida no brasil; ou o etanol produzido a partir de milho, como nos estados Unidos; ou o metano advindo da biomassa. \source
% Esses combustíveis reduzem  ...

% Exemplo de citação \cite{Wu2023}.

\subsection{Objetivo}




% \subsection{Motivação}
% \subsection{Lacunas}
% \subsection{Contribuição Científica}
% \subsection{Objetivos}

Desenvolver um modelo 