\section{Introdução} \label{sec:intro}

% 1. Introdução
%    1. Motivação
%       1. entre as duas conectar c como e pq simular
%    2. Lacunas (Revisão Bibliográfica, estado da arte)
%    3. Justificativa
%    4. Objetivos (e Hipótese)
%    5. Contribuição Científica

% Importância de trabalhar com combustão no contexto global.
% Como se simula combustão
% Necessidade de modelos HMT para partículas em simulações de combustão e sua importância

% Explain need for multicomponent modeling in current global context
% Explain need for modeling the interior of the droplet
% Explain need for droplet combustion model

% GAP: lack of models for single droplet combustion, modeling way behind pure evaporation
% GAP: lack of consistent multicomponent version of evaporation models and lack of multicomponent single-droplet burning models.
% GAP: lack of computationally affordable methods that account for droplet interior AND advanced evaporation/SDC models.

A demanda pela transição energética e pela descarbonização da economia busca alternativas para substituição dos combustíveis fósseis nos setores de energia, transporte, indústria. 
Alguns setores, como o energético e o de transportes urbano de baixa carga, têm mostrado grande progresso no uso de energias renováveis e na eletrificação, respectivamente \source{}. 
Porém, combustíveis fósseis são extremamente difíceis de substituir em outros setores, especialmente os combustíveis líquidos.
São estes os que possuem maior energia específica e densidade energética \source{Reviews de Al}, sendo os mais adequados para aplicações de transporte, como no setor automotivo de cargas pesadas, naval e aeronáutico \cite{MasriA2021}.
Além do setor de transporte, combustíveis líquidos são importantes para algumas indústrias como aço e cimento, para algumas termoelétricas e até para máquinas de pequeno porte, portáteis, movidas a motor de combustão interna (MCI).
Por fim, uma análise histórica indica que a transição para fontes renováveis se dará ao longo de décadas \cite{MasriA2021}.

% São exemplos de combustíveis líquidos a gasolina, o etanol, o diesel, o kerosene, o kerosene de aviação, etc. 
% O etanol, em particular, é de grande relevância para o Brasil.

Nota-se que processos de combustão continuarão relevates nas próximas décadas.
Em especial, todas as aplicações mencionadas baseam-se na combustão turbulenta de sprays líquidos.
Assim, soluções devem ser procuradas para concicliar essa tecnologia com os esforços de transição energética e descarbonização da economia.
A comunidade científica busca, então, três caminhos: \textbf{(i)} novos combustíveis; \textbf{(ii)} novas origens para os mesmos combustíveis; \textbf{(iii)} melhorar a eficiência dos motores a combustão e reduzir a formação de poluentes.

Independente da origem do combustível, o processo de combustão deve ser compreendido para que motores e queimadores eficientes e com baixa emissão de poluentes sejam desenvolvidos.
Para tanto é necessário pesquisa em combustão, que pode ser estruturada em trabalhos experimentais ou trabalhos de modelagem.
A modelagem da combustão turbulenta de sprays, foco desse trabalho, deve ser capaz de representar diferentes combustíveis líquidos, incluindo combustíveis oriundos das demandas (i) e (ii).
Deve também representar os diferentes fenômenos envolvidos nesse processo, como ignição e formação de poluentes, de forma a atender a demanda (iii).
No âmbito da combustão turbulenta de sprays, é de extrema importância o modelo de transferência de calor e massa da gota (líquida) para a fase gasosa, ou seja, o modelo de evaporação e condensação das gotas do spray.
É conhecido que essa modelagem tem enorme influência na chama como um todo \cite{JennyB2012}, influenciando a sua estrutura, temperatura, geometria e, por consequência, formação de poluentes também.

Nesse sentido, revisando a literatura mais recente, nota-se a necessidade de maior desenvolvimento desses modelos de evaporação e condensação (MEC) para representar corretamente diferentes combustíveis.
Por exemplo, modelos monocomponentes com equilíbrio termodinâmico não são capazes de representar todos os fenômenos associados, por exemplo, à combustão de etanol.
Em especial, nota-se uma demanda pela aplicação de modelos complexos, desenvolvidos e testados na escala de uma única gota, em simulações de larga escala da chama como um todo.
Para representar combustíveis como o etanol, anidro ou hidratado, é necessário uma modelar a gota de forma \textbf{multicomponente}, considerando \textbf{termodinâmica de mistura não-ideal} e os efeitos de transferência de calor e massa no \textbf{interior da gota}.

No modelo HMT descrito acima, a reação química é resolvida separadamente do modelo da gota.
Isso corresponde a uma chama longe e não presa a gota, em que a chama é alimentada pelo vapor de combustível oriundo de várias gotas evaporando. 
Em contraste, é possível que ocorra a combustão de uma gota isolada, com uma frente de chama próxima e circundante à gota, a uma distância na mesma ordem de grandeza que o diâmetro da gota.
Isso é chamado \textbf{combustão de gota isolada}.
Revisando a literatura, constatou-se que não é muito claro quando a combustão de gota isolada ocorre \source{}, nem o seu efeito na chama em larga escala.
Sabe-se que a combustão de gota isolada é importante para a ignição do spray \cite{AggarwalS2014}, mas não foram encontrados estudos sobre o seu impacto na estrutura da chama.

\subsection{Objetivos} \label{sec:objetivos}

Com isso em vista, a presente tese visa estudar e desenvolver modelos analíticos para a transferência de calor e masssa em gotas em dois cenários: 
\begin{enumerate}
    \item[\textbf{A.}] Modelo de Evaporaçào e Condensação (MEC); e 
    \item[\textbf{B.}] Modelo de Combustão de Gota Isolada (MCDI).
\end{enumerate}
Esses modelos devem considerar os seguintes aspectos: 
\begin{enumerate}
    \item[\textbf{1.}] Descrição multicomponente da gota; 
    \item[\textbf{2.}] Mistura com termodinâmica não-ideal; 
    \item[\textbf{3.}] Efeitos de transferência	de calor e massa no interior da gota. 
\end{enumerate}
O objetivo é desenvolver simulações de larga escala com ambos modelos, {A} e {B}, com a capacidade:
\begin{enumerate}
    \item[\textbf{4.}] Determinação de quando ocorre a combustão de gota isolada;
\end{enumerate}
ou seja, de determinar se utiliza modelo {A} ou o modelo {B}.
Dessa forma, o autor visa investigar o efeito de cosiderar a combustao de gota isolada, e dos aspectos {1}, {2} e {3}, na estrutura da chama.
A investigação da estrutura da chama inclui a investigação de aspectos como destribuição de temperatura e concentraçoes de espécies, o que é essencial para a determinação da ignição, da eficiência de combustão e das emissões de poluentes.


% A modelagem de combustão de gota isolada (MCGI), reduz a complexidade do problema de cinética química da simulação, mas aumenta a complexidade do modelo de transferência de calor e massa da gota, quando comparado ao modelo de evaporação/condensação (MEC).
% Assim, faz sentido desenvolver MEC primeiro, e depois o MCGI.
% Também, considerando que o objetivo é realizar uma simulação de chama turbulenta que use tanto um MEC quanto de um MCGI, é desejável que ambos tenhas as mesmas capacidades, (1) a (3).
% Por esse motivo, e considerando que, para simular, uma chama turbulenta de spray é necessário tanto um MEC quanto um MCGI,

% Utilizando essa abordagem, obtém-se, em larga escala, chamas ao redor de um grupo de gotas, internas ou exernas ao grupo (\emph{internal group combustion} ou \emph{external group combustion}), ou frentes de chama externas e não envovlventes (\emph{external sheath combustion}).
% Porém, uma frente chama também pode se formar ao redor de uma única gota, condição chamada combustão de gota isolada (\emph{single droplet combustion}).
% Revisando literatura, nota-se que não é claro quando a combustão de spray ocorre em cada modo \source



% São escoamentos particulados, dispersos e reativos, que envolvem fenômenos multi-escala, multifásicos e, frequentemente, combustíveis multicomponente.
% Revisando a literatura mais recente, o autor notou a ausência de estudos investigando a influência de modelar a combustão de gota isolada na estrutura da chama.
% Ademais, há um atraso em termos de complexidade na utilização de modelos de transferência de massa e calor (HMT - \emph{heat and mass transder}) em gotas nas simulações de larga escala, em comparação com simulações de uma única gota.
% Revisando a literatura mais recente, constatou-se uma dificuldade de identificar e modelar diferentes modos de combustão de spray.
% \colorbox{cyan}{Em particular, como a combustão isolada de gotículas afeta a estrutura da chama.}

% Na área da modelagem, que é foco dessa pesquisa
% Como se entende o processo ... experimento e modelagem ... foco em modelagem e simulação ... CFD.
% A simulação de combustão gasosa já é bem complexa, em parte devido à coexistência de fenômenos químicos e fenômenos de transporte, assim pela influência da turbulência e as interações entre esses aspectos.
% A combustão de sprays traz a complexidade adicional de possuir duas fases: uma fase contínua e gasosa, e uma fase dispersa, geralmente líquida.
% Escoamentos bifásicos como no caso de aerosóis ou sprays já são bem complexos, mesmo quando não reativos.
% Talvez por isso, para a simulação de escoamentos reativos de sprays, para parte química, utiliza-se técnicas desenvolvidas para a combustão gasosa.
% Combustíveis ... a maioria é líquido ... maior complexidade e desafios ...
% Foco em combustão de sprays líquidos ... 

% A modelagem para a sinulação de escoamentos reativos, dispersos e multifásicos, como na combustão de sprays, requer três partes: (i) a modelagem da fase contínua gasosa; (ii) a modelagem da fase dispersa, líquida; (iii) a modelagem da química homogênea, i.e. da combustão na fase gasosa.
% Para a parte (iii), a maioria dos trabalhos de combustão de sprays utiliza metodologias desenvolvidas para a combustão gasosa, considerando as gotas apenas como fontes de vapor de combustível.
% Isso corresponde 

% Necessidade e dificuldade de simular a combustão turbulenta de Sprays... Mencionar revisões ... que tipos de simulação tem .. dificuldade e necessidade de simulações de larga escala ... normalmente com química tabelada.

% Âmbito e objetivo desse trabalho.

% Necessidade e dificuldade de representação multicomponente da gota.

% Necessidade e dificuldade de representar o interior da gota.

% Necessidade e dificuldade de representar diferentes modos de combustão de gota -> combustão de gota isolada.

% No cenario atual de busca pela transição energética e descarbonização, a pesquisa em combustão é mais importânte do que nunca.
% Nesse contexto, os principais objetivos são reduzir emissões, adaptar os modelos para novos combustíveis mais limpos e para novos modos de combustão, necessários para motores mais limpos.

% Análises históricas indicam que a transição para fontes de energia renováveis se dará ao longo de décadas \cite{MasriA2021}, [MASRI 1, 3 também]. Até lá, combustíveis fosseis mantém o seu domínio.
% Além disso, no setor de transportes, a eletrificação dificilmente substituirá combustíveis fósseis em veículos de cargas pesadas ou para transporte aéreo ou marítmo, mesmo considerando previsões otimistas para o desevolvimento de baterias \cite{MasriA2021}. O mesmo pode ser dito para processos industriais intensivos em energia, como cimento e aço.

% Enquanto os caminhos (i) e (ii) concerne principalmente, aos químicos, o caminho (iii) requer a modelagem de processos de combustão. 
% A primeria abordagem consiste em desevolver o uso de combustíveis que não poluam -- como hidrogênio, amônia e pós metálicos -- ou que poluam menos -- como OMEs e naphta (ver \cite{MasriA2021}) -- ou que venham de outras fontes -- como o etanol e o metanol.
% Dentre as fontes não fósseis, se destacam os bio-combustíveis, produzidos a partir de biomassa renovável, e o eletrocombustíveis, sintetizados a partir de água, gás carbônico e energia de fontes renováveis.
% Em particular o etanol é importante para o Brasil ... \yellow{TODO}
% São exemplos da primeira abordagem o interesse crescente na combustão da amônia \source{}, do hidrogênio \source{}, do metanol \source{} e do etanol \source{} -- conhecidos Brasil --, de pós metálicos \source{} e outros \source{}. 
% Pela segunda abordagem, se destacam os eletrocombustíveis (\emph{e-fuels}), sintetizados a partir de gás carbônico, água e energia elétrica, e os bio-combustíveis.

% Uma opçao para a descarbonização desses setores é o uso de eletrocombustíveis, também chamados de \emph{e-fuels}, \emph{power-to-x} (PtX) ou \emph{powerfuels}.
% Estes são combustíveis líquidos ou gasosos produzidos a partir de água, gas carbônico ($\text{CO}_2$), eventualmente nitrogênio $\text N_2$ e energias de fontes renováveis. 
% Pode-se produzir, por exemplo: hidrogênio, metano, metanol, hidrocarbonetos líquidos (processo Fischer-Tropf), éteres de oximetileno (OMEs) e amônia (processo Haber-Bosch). source{MASRI 9}.
% Os combustíveis PtX podem utilizar a infraestrutura de transporte, distribuição e armazenamento já existente dos combustíveis fósseis \source{MASRI 9-13} e podem chegar a ser neutros em termos de emissões, apesar que ainda mais caros que combustíveis de origem fóssil \source{MASRI 42, 43}. 

% Outra alternativa para combustíveis de origem fóssil são os combustíveis verdes ou bio-combustíveis, produzidos a partir de platas. 
% Por exemplo, o etanol produzidos a partir do bagaço de cana, tecnologia desenvolvida no brasil; ou o etanol produzido a partir de milho, como nos estados Unidos; ou o metano advindo da biomassa. \source
% Esses combustíveis reduzem  ...

% Exemplo de citação \cite{Wu2023}.


% \subsection{Motivação}
% \subsection{Lacunas}
% \subsection{Contribuição Científica}
% \subsection{Objetivos}

% Desenvolver um modelo 