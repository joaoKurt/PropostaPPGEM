\section{Introdução}


No cenario atual de busca pela transição energética e descarbonização, a pesquisa em combustão é mais importânte do que nunca.
Nesse contexto, os principais objetivos são reduzir emissões, adaptar os modelos para novos combustíveis mais limpos e para novos modos de combustão, necessários para motores mais limpos.

Análises históricas indicam que a transição para fontes de energia renováveis se dará ao longo de décadas \cite{MasriA2021}, [MASRI 1, 3 também]. Até lá, combustíveis fosseis mantém o seu domínio.
Além disso, no setor de transportes, a eletrificação dificilmente substituirá combustíveis fósseis em veículos de cargas pesadas ou para transporte aéreo ou marítmo, mesmo considerando previsões otimistas para o desevolvimento de baterias \cite{MasriA2021}. O mesmo pode ser dito para processos industriais intensivos em energia, como cimento e aço.

Uma opçao para a descarbonização desses setores é o uso de eletrocombustíveis, também chamados de \emph{e-fuels}, \emph{power-to-x} (PtX) ou \emph{powerfuels}.
Estes são combustíveis líquidos ou gasosos produzidos a partir de água, gas carbônico ($\text{CO}_2$), eventualmente nitrogênio $\text N_2$ e energias de fontes renováveis. 
Pode-se produzir, por exemplo: hidrogênio, metano, metanol, hidrocarbonetos líquidos (processo Fischer-Tropf), éteres de oximetileno (OMEs) e amônia (processo Haber-Bosch). source{MASRI 9}.
Os combustíveis PtX podem utilizar a infraestrutura de transporte, distribuição e armazenamento já existente dos combustíveis fósseis \source{MASRI 9-13} e podem chegar a ser neutros em termos de emissões, apesar que ainda mais caros que combustíveis de origem fóssil \source{MASRI 42, 43}. 

Outra alternativa para combustíveis de origem fóssil são os combustíveis verdes ou bio-combustíveis, produzidos a partir de platas. 
Por exemplo, o etanol e o metanol produzidos a partir do bagaço de cana, tecnologia desenvolvida no brasil; ou o etanol produzido a partir de milho, como nos estados Unidos; ou o metano advindo da biomassa. \source
Esses combustíveis reduzem  ...

Outra alternativa ainda é o uso de pós metálicos...

% A eletrólise da água produz hidrogênio, $\text H_2$.
% Hidrogênio com gas carbônico pode produzir hidrocarbonetos a partir do processo Fischer-Tropsch, ou metano, metanol ou éteres de oximetileno (OMEs).
% Por outro lado, hidrogênio também pode ser utilizado diretamente em células de combustível ou em motores de combustão futuros, ou produzir amônmia a partir do processo de Haber-Bosch.  



% Necessidade de trabalhar com combustão no contexto global.

% Necessidade de modelos HMT para partículas em simulações de combustão e sua importância


% Explain need for multicomponent modeling in current global context
% Explain need for modeling the interior of the droplet
% Explain need for droplet combustion model

% GAP: lack of models for single droplet combustion, modeling way behind pure evaporation
% GAP: lack of consistent multicomponent version of evaporation models and lack of multicomponent single-droplet burning models.
% GAP: lack of computationally affordable methods that account for droplet interior AND advanced evaporation/SDC models.

Exemplo de citação \cite{Wu2023}.





% \subsection{Motivação}
% \subsection{Lacunas}
% \subsection{Contribuição Científica}

\subsection{Objetivos}

Desenvolver um modelo 