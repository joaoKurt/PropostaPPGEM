% !TEX root = ../Proposta.tex

\section{Forma de Análise dos Resultados} \label{sec:resultados}

Diferentes procedimentos serão adotados para a análise dos resultados obtidos em cada etapa de trabalho apresentada na Seção \ref{sec:metod}.
Considerando apenas os resultados numéricos, i.e. das de simulações, os primeiros resultados serão obtidos após o teste do modelo isolado.

A simulação da evolução temporal de um modelo de gota, MEC ou MCGI, e a análise desses resultados é uma área que o autor da proposta já tem experiência \cite{HenningsJ2024MT}. 
Nessa análise, são relevantes parâmetros como o tempo de vida da gota, a dependência dos resultados de condições iniciais da gota e do ambiente e a dependência dos submodelos utilizados, como o de pressão de vapor ou de fração molar de vapor.

No contexto de avaliação isolada de MECs e MCGIs, é de extrema relevância a comparação com resultados tanto experimentais e quanto de estudos numéricos detalhados.
Para MECs, consideram-se, por exemplo, os trabalhos experimentais \cite{BiroukM2006,PatelU2019,KayaEyiceD2024,ArabkhalajA2024,MaquaC2008}, focados na influência da turbulência na evaporação.
Para MCGIs, consideram-se os trabalhos experimentais na escala da gota \cite{ChoS1990SCI,CandelS1999,ChenG1996CF,Xu2002,BiroukM2000,CuociA2005,SetyawanH2015} e os trabalhos baseados em DNS que resolvem o modelo da gota \cite{Stauch2006,CuociA2005,ChoS1990SCI,KazakovA2003CF,MarcheseA1996CF,WangW2024}.

Já para as simulações em ambiente simplificado, a análise dos resultados se dará baseada na experiência do grupo de pesquisa, assim como na comparação com MECs já desenvolvidos pelo grupo e testados nesse ambiente \cite{SacomanoF2018CTM,SacomanoF2019IJHMT,SacomanoF2021Fluids,SacomanoF2024CF,SacomanoF2025CF}.
Nessas simulações, são relevantes a influência das condições iniciais e ambientais na frente de chama, em particular na velocidade de chama laminar.
Esse ambiente é muito importante para análises paramétricas e preliminares.

Nas simulações multidimensionais, também será valiosa a experiência e os resultados anteriores do grupo de pesquisa \cite{SacomanoF2017CF,SacomanoF2020CF}.
Elas possibilitarão estudar a influência do modo de combustão de gota isolada na estrutura da chama, como proposto nos objetivos do trabalho.