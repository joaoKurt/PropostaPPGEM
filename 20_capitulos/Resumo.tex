% !TeX root = ..\Proposta.tex

\vspace{2cm}

{ \Large \textbf{Resumo}}

\vspace{0.8cm}

{
\setstretch{1}

\noindent
A modelagem de chamas turbulentas de sprays diluídos geralmente é modelada utilizando modelos de evaporação e condensação, o que reproduz uma frente de chama externa às gotas.
No entanto, chamas envelopadas ao redor de gotas isoladas já foram observadas em experimentos e simulações de chamas turbulentas de sprays líquidos.
Também denominadas combustão de gotas isoladas, essas chamas já foram relacionadas ao processo de ignição de sprays e à formação de fuligem.
Revisando a literatura, constatou-se a necessidade de incluir a modelagem de chamas envelopadas em simulações turbulentas de sprays diluídos de combustíveis líquidos.
O desenvolvimento dessa modelagem deve representar efeitos de combustíveis reais, que são misturas de espécies químicas, como o querosene, e/ou hidrofílicos, como o etanol.
Para tanto devem incluir a capacidade de representação de uma mistura multicomponente, a consideração de fenômenos de transporte no interior da gota e a consideração de termodinâmica de mistura não ideal.
A construção de modelos de combustão de gota isolada (MCGI) passará pela construção de modelos de evaporação e condensação (MEC) com as mesmas funcionalidades, garantindo assim um aumento gradual da sofisticação dos modelos desenvolvidos e baseando-se sempre nos modelos já existentes na literatura.
A co-utilização de MEC e MCGI em uma simulação deve ser acompanhada por um mecanismo de seleção, também a ser desenvolvido, que escolherá qual modelo utilizar para cada gota.
O novo modelo será testado primeiro isoladamente, para garantir o correto funcionamento e implementação. 
Em seguida, será testado no cenário simplificado de uma chama laminar de névoa quiescente de spray líquido monodisperso.
Por fim, após aprofundamento em interação chama-turbulência, o modelo será utilizado em simulações multidimensionais turbulentas.

}
\vspace{1cm}

{ \Large \textbf{Abstract}}

\vspace{0.8cm}

{
\setstretch{1}

\noindent
The modeling of turbulent diluted spray flame is usually modelled using evaporation and condensation models, which produces an external flame sheet.
Nonetheless, enveloped flames around isolated droplets have already been seen in experiments and simulations of turbulent liquid spray flames.
Also named single or isolated droplet combustion, these flames have already been related to spray ignition processes and formation of soot.
Reviewing the literature, it was clear that the modeling of enveloped flames should be included in simulations of turbulent diluted liquid spray flames. 
The development of such models ought to represent real fuels which are mixtures of diferent chemical species, like kerosene, and/or hydrophilic, like ethanol.  
Therefore they must be able to represent multicomponent mixtures, to consider transport phenomena inside the droplet and to consider non-ideal mixture thermodynamics.
Developing single droplet combustion models (MCGI) will encompass the developement of evaporation and condensation models (MEC) with the same capabilities, thus ensuring a gradual model development process, which will be based on pre-existing models in the literature.
The simultaneous use of MEC and MCGi in a simulation must include a switch, also to be developed, that will chose which model to apply to each droplet.
The new model will first be tested independently, to ensure its correct working and implementation.
Then, it will be tested in the simplified cenario of a laminar flame in a quiescent monodisperse droplet mist.
Lastly, after investigating flame-turbulence interaction, the model will be used in multidimensionais turbulent spray flame simulations.

}