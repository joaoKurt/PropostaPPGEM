% !TeX root = ..\Proposta.tex

\vspace{2cm}

{ \Large \textbf{Resumo}}

\vspace{0.8cm}

{
\setstretch{1}

\noindent % Se mudar aqui, lembrar de mudar no SAGE também!
A combustão turbulenta de spray é um processo essencial em diversas tecnologias. 
Sua modelagem e simulação fazem parte do esforço atual de transição energética e descarbonização.
Simulações de combustão de sprays diluídos, baseadas na dinâmica dos fluidos computacional, geralmente utilizam modelos de evaporação e condensação para as gotas.
Estes descrevem uma frente de chama externa às gotas, estabilizada pela mistura vapor-oxidante formada pela evaporação do combustível.
No entanto, chamas estabilizadas ao redor de gotas individuais também são observadas em experimentos e simulações.
Também denominadas combustão de gotas isoladas, essas chamas estão relacionadas ao processo de ignição de sprays e à formação de fuligem.
No entanto, a sua modelagem é raramente incluída em simulações computacionais.
Este trabalho visa desenvolver modelos de combustão de gota isolada (MCGI) e incluí-los em simulações de combustão turbulenta multidimensional para investigar a sua influência na combustão de sprays.
Esse desenvolvimento inclui a elaboração de modelos de evaporação e condensação (MEC).
Ambos devem ser capazes de representar efeitos de combustíveis comerciais, que são multicomponentes (como a gasolina) e/ou hidrofílicos (como o etanol e o metanol). 
Para tanto, devem ser considerados também fenômenos de transporte no interior da gota e termodinâmica de mistura não ideal.
Dessa forma, um segundo objetivo deste trabalho é avaliar os impactos do aumento de capacidade descritiva de modelos de transferência de calor e massa (MEC e MCGI) na combustão de sprays multicomponentes.
A co-utilização de MEC e MCGI em uma simulação, como proposto, deve ser acompanhada por um mecanismo de seleção de modos de combustão, a ser desenvolvido, que escolherá qual modelo utilizar para cada gota.
Esse trabalho contribuirá para o aperfeiçoamento da capacidade preditiva de simulações de chama turbulenta multidimensional de sprays multicomponentes.
% Os modelos desenvolvidos serão testados isoladamente e em uma chama laminar de névoa de spray, quiescente e monodispersa, antes de serem utilizados em simulações de chamas multidimensionais turbulentas. 
% #DONE Motivação
% #DONE Objetivo
% Primeiramente, o novo modelo será testado de forma isolada, para garantir o correto funcionamento e implementação. 
% Em seguida, será testado no cenário simplificado de uma chama laminar de névoa quiescente de spray líquido monodisperso.
% Por fim, após aprofundamento em interação chama-turbulência, o modelo será utilizado em simulações multidimensionais turbulentas.
% # DONE Resultado e contribuição para comunidade científica
% Sugestão Sacomano:
%  O desenvolvimento desse trabalho deve contribuir para o aperfeiçoamento da capacidade preditiva de ... por programas baseados na dinâmica dos fluidos computacional (CFD).

}
\vspace{1cm}

{ \Large \textbf{Abstract}}

\vspace{0.8cm}

{
\setstretch{1}

% #TODO Update translation
\noindent % Se mudar aqui, lembrar de mudar no SAGE também!
The modeling of turbulent diluted spray flame is usually modelled using evaporation and condensation models, which produces an external flame sheet.
Nonetheless, enveloped flames around isolated droplets have already been seen in experiments and simulations of turbulent liquid spray flames.
Also named single or isolated droplet combustion, these flames have already been related to spray ignition processes and formation of soot.
Reviewing the literature, it was clear that the modeling of enveloped flames should be included in simulations of turbulent diluted liquid spray flames. 
The development of such models ought to represent real fuels which are mixtures of diferent chemical species, like kerosene, and/or hydrophilic, like ethanol.  
Therefore they must be able to represent multicomponent mixtures, to consider transport phenomena inside the droplet and to consider non-ideal mixture thermodynamics.
Developing single droplet combustion models (MCGI) will encompass the developement of evaporation and condensation models (MEC) with the same capabilities, thus ensuring a gradual model development process, which will be based on pre-existing models in the literature.
The simultaneous use of MEC and MCGi in a simulation must include a switch, also to be developed, that will chose which model to apply to each droplet.
The new model will first be tested independently, to ensure its correct working and implementation.
Then, it will be tested in the simplified cenario of a laminar flame in a quiescent monodisperse droplet mist.
Lastly, after investigating flame-turbulence interaction, the model will be used in multidimensionais turbulent spray flame simulations.

}