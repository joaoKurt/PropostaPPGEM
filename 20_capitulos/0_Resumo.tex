% !TeX root = ..\Proposta.tex

\vspace{2cm}

{ \Large \textbf{Resumo}}

\vspace{0.8cm}

{
\setstretch{1}
% \linenumbers % Remover

\noindent % Se mudar aqui, lembrar de mudar no SAGE também!
A combustão turbulenta de spray é um processo comumente encontrado em diversas tecnologias que atendem diversos setores econômicos. 
O aperfeiçoamento de sua modelagem e simulação faz parte do esforço atual para transição energética e descarbonização.
Simulações de combustão de sprays diluídos, baseadas na dinâmica dos fluidos computacional, geralmente utilizam modelos de evaporação e condensação (MECs) para as gotas, os quais descrevem uma frente de chama externa às gotas.
No entanto, chamas estabilizadas ao redor de gotas individuais também são observadas em experimentos e simulações.
Denominadas combustão de gota isolada, essas chamas estão relacionadas ao processo de ignição de sprays e à formação de fuligem.
Contudo, a sua modelagem é raramente incluída em simulações computacionais.
Este trabalho visa desenvolver modelos de combustão de gota isolada (MCGI) e incluí-los em simulações de chamas turbulentas para investigar a sua influência na combustão de sprays.
Esse desenvolvimento inclui a elaboração de MECs.
Ambos modelos devem representar efeitos de combustíveis comerciais, os quais são multicomponentes (como a gasolina) e/ou hidrofílicos (como o etanol, o metanol e a amônia). 
Para tanto, devem ser considerados também fenômenos de transporte no interior da gota e termodinâmica de mistura não ideal.
Dessa forma, um segundo objetivo deste trabalho é avaliar os impactos do aumento de capacidade descritiva de MECs e MCGIs na combustão de sprays.
A consideração de ambos modos de combustão, de gota isolada e externa, requer o desenvolvimento de um mecanismo de que determine qual cenário considerar em cada gota.
Esse trabalho contribuirá para o aperfeiçoamento da capacidade preditiva de simulações de chama turbulenta de sprays multicomponentes.

}

\vfil

{ \Large \textbf{Abstract}}

\vspace{0.8cm}

{
\setstretch{1}
% \resetlinenumber \linenumbers % Remover

% #TODO Update translation
\noindent % Se mudar aqui, lembrar de mudar no SAGE também!
The modeling of turbulent diluted spray flame is usually modelled using evaporation and condensation models, which produces an external flame sheet.
Nonetheless, enveloped flames around isolated droplets have already been seen in experiments and simulations of turbulent liquid spray flames.
Also named single or isolated droplet combustion, these flames have already been related to spray ignition processes and formation of soot. Reviewing the literature, it was clear that the modeling of enveloped flames should be included in simulations of turbulent diluted liquid spray flames. The development of such models ought to represent real fuels which are mixtures of different chemical species, like kerosene, and/or hydrophilic, like ethanol.
Therefore they must be able to represent multicomponent mixtures, to consider transport phenomena inside the droplet and to consider non-ideal mixture thermodynamics.
Developing single droplet combustion models (MCGI) will encompass the development of evaporation and condensation models (MEC) with the same capabilities, thus ensuring a gradual model development process, which will be based on pre-existing models in the literature.
The simultaneous use of MEC and MCGI in a simulation must include a switch, also to be developed, that will chose which model to apply to each droplet.
The new model will first be tested independently, to ensure its correct working and implementation.
Then, it will be tested in the simplified scenario of a laminar flame in a quiescent monodisperse droplet mist.
Lastly, after investigating flame-turbulence interaction, the model will be used in multidimensional turbulent spray flame simulations.

}