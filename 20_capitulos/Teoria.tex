\section{Fundamentação Teórica}

% 2. Fundamentação Teórica (meio q Trabalhos preliminares)
%    1. Transferencia de calor e massa em gotículas (experiencia do grupo)
%       1. Multicomponente
%       2. Interior de gota
%    2. Single Droplet Burning (minha exp)
%       1. Monocomponente (contexto de sprays e meu trabalho)
%       2. Multicomponente (Sekularac e ArabkhalajA2024)
%       3. Probabilidade de acontecer
%    3. Turbulent spray combustion
%       1. Modelagem mecflu
%          1. LES, eqs de transporte, ATF
%       2. Modelagem da química
%          1. FGM


\subsection{Combustão Turbulenta de Sprays}

A combustão turbulenta de sprays é caracterizada pela competição de vários processos físicos e químicos, fortemente acoplados e em diferentes escalas de tempo e comprimento. 
Na formação de um spray turbulento, um jato de combustível líquido se quebra devido às instabiliades hidrodinâmica de Kevin-Helmholz e Rayleigh-Taylor, formando gotas que se dispersam, deformam e atomizam devido às forças aerodinâmicas superando as tensões superficiais da gota \cite{JennyB2012}.
Isso forma o {regime denso} do spray, onde ocorrem também outros fenômenos como colisões, coalescência e interferência por esteira aerodinâmica, por turbulência ou por alteração da concentração de vapor de combustível devido à evaporação.
A medida que o jato se atomiza em gotas menores e dispersas, as gotas deixam de interferir umas nas outras e o regime é chamado de disperso ou diluído. 
Desde a sua formação, as gotas de combustível evaporam, fornecendo vapor combustível para a chama, que por sua vez influencia e é influenciada pelas próprias gotas e pela turbulência local.
Revisões detalhadas e com mais referências para processos e interações na combustão turbulenta de sprays podems er encontradas em \cite{JennyB2012, MasriA2016, SanchezA2015, ZhouL2021}.

O foco deste trabalho é na modelagem das escalas da gota (micro) e do spray e da chama (macro) na região diluida de um spray de combustível líquido.
Modelar a escala macro requer um modelo para a fase contínua, gasosa, um modelo para as reações químicas e um modelo para a fase dispersa, as gotas.
Dois exemplos de modelos para a fase contínua e para a química, escolhidos por relevância e experiência no grupo de pesquisa, serão apresentadas nas próximas Subseções \ref{sec:gas} e \ref{sec:chem}.
Atenção especial será dada para a modelagem da gota, uma vez que o foco desse trabalho é desenvolver novos modelos para essa escala e investigar os efeitos na escala da chama.
Modelos de transferência de calor e massa em gotas são discutidos nas Seções \ref{sec:MEC}. 

% Na abordagem macro, há duas fases nesse sistema: a fase contínua (gasosa) e a fase líquida (gotas); e ambas precisam de modelagem termo-física e química.
% Para a fase contínua, gasosa, isso se traduz nas equações de transporte, adequadas ao modelo de turbulência escolhido, e em uma modelagem química, que serão resolvidas durante a simulação.

% Já a fase contínua, por ser na escala da partícula, precisa ser modelada com um modelo analítico, já que a 

% Para a modelagem macro, a modelagem micro precisa ser incluida.
% Nas simulações Euler-Lagrange, foco deste trabalho e deste grupo de pesquisa,  

% A medida que as gotas se formam, elas evaporam, alterando a concentração
% A combustão turbulenta de sprays é um problema multi-fásico, multi-componente, multi-escala, multi-região e multi-físico. 

\subsubsection{Modelagem da Fase Contínua} \label{sec:gas}

As equações que governam a fase contínua são as equações de conservação de espécie, quantidade de movimento e energia, junto com as condições de contorno para a interface gás-líquido e relações termodinâmicas necessárias para o fechamento do problema.
Derivações desse conjunto de equações podem ser encontradas por exemplo nos livro-texto \cite{Williams1985,Kuo2005,Law2006}.

Para as simulações na escala da chama de spray, em que não é viável resolver a interface das gotas, dado que elas estão em outra escala de comprimento, muito menor que a chama, as gotas são representadas como pontos infinitamente pequenos.
A sua influência na fase contínua se dá, então, através de termos fonte obtidos a partir de modelos analíticos na escala da gota. 
Essa abordagem é denominada \emph{PSIC - Particle Source in Cell} e a aproximação feita para a partícula é chamada de \emph{point particle approximation}.
As equações de transporte resultantes estão descritas abaixo.

\begin{equation}
    \frac{\partial \rho Y^\beta}{\partial t} + 
    \frac{\partial \rho Y^\beta U_j}{\partial x_j} =
    - \frac{\partial J_j^\beta}{\partial x_j} +
    S^\beta -
    \dot \rho^{\beta^D}
\end{equation}
\begin{equation}
    \frac{\partial \rho U_i}{\partial t} + 
    \frac{\partial \rho U_i U_j}{\partial x_j} =
    - \frac{\partial p}{\partial x_i} +
    \frac{\partial \tau_{ij}}{\partial x_j} -
    \rho g\frac{\partial z}{\partial x_i} -
    \dot M_i^D -
    f_i^D
\end{equation}
\begin{equation}
    \frac{\partial \rho E}{\partial t} + 
    \frac{\partial \rho H U_j}{\partial x_j} =
    \frac{\partial J_j^h}{\partial x_j} +
    \rho U_i g \frac{\partial z}{\partial x_i} +
    Q -
    \dot Q^D -
    \dot q^D
\end{equation}

Essas equações de transporte são respectivamente da fração mássica da espécie $\beta$, $Y^\beta$, da quantidade de momento na direção $i$, $\rho U_i$ e da energia $E=e+(1/2)U_iU_i+zg$. 
Os termos de interação entre fases são aqueles com o sobrescrito $D$ (de \emph{droplet} - gota).
Com excessão dos termos, $\dot M_i^D$ e $f_i^D$, que correspondem às trocas de momentum e força, os termos restantes, $\dot \rho^{\beta^D}$, $\dot Q^D$ e $\dot q^D$ são oriundos dos modelos de transferência de calor e massa, detalhados na Seção \ref{sec:MEC} e \ref{sec:MCGI}. 
A notação segue o padrão utilizado por \cite{JennyB2012}, consulte-o para mais explicações.
% São esses termos que precisam ser modelados pelos modelos de gota, descritos na Seção
% $J_j^\beta$ é o fluxo de massa especie

O groupo de pesquisa tem experiência com simulações de vórtices de larga escala (LES - \emph{Large Eddy Simulations}), nas quais as equações de transporte são decompostas entre sub-escala e escala, $\psi = \widetilde\psi + \psi^"$, e filtradas de acordo com $\widetilde\psi = \overline{\rho\psi}/\overline\rho$, em que $\psi$ corresponde a uma variável qualquer. 
Em uma formulação de densidade variável para números de Mach baixos, as equações de continuidade e de quantidade de movimento são
\begin{equation}
\frac{\partial \bar \rho}{\partial t} + 
\frac{\partial \bar \rho \widetilde u_i}{\partial x_i} = 
\overline S_v
\end{equation}
\begin{equation}
\frac{\partial \bar\rho \widetilde u_i}{\partial t} + 
\frac{\partial \bar\rho \widetilde u_i \widetilde u_j}{\partial x_j} =
\frac{\partial }{\partial x_j} \left(
	2\bar\mu \widetilde S_{ij} -
	\frac{2}{3}\bar\mu \frac{\partial \widetilde u_k}{\partial x_k} \delta_{ij} -
	\bar\rho \tau_{ij}^{\text{sgs}}
\right) -
\frac{\partial \bar p}{\partial x_i} +
\bar p g_i + 
\overline S_{u,i}
\end{equation}
em que os termos $\overline S_v$ e $\overline S_{u,i}$ são os termos de acoplamento de fase de massa e de momento. Essa formulação foi utilizada em \cite{SacomanoF2017PhD, SacomanoF2017CF, SacomanoF2020CF} junto com a abordagem química FGM (\emph{Flamelet Generated Manifold}), explicada na próxima Seção, no desenvolvimento de um método de espessamento de chama dinâmico (ATF - \emph{Artificially Thickened Flame}).
\todo{Explicar o que cada um fez, baseado no áudio. Fazer? }.


Essas são as simulações mais completas de combustão turbulenta de sprays, que podem ser utilizadas em cenário reais, inclusive para simular cenários realizados também em experimentos, para validação \source{}.
Entretando, em alguns aspectos é relevate realizar simulações laminares em configurações mais simples, canônicas, para investigar alguns aspectos da modelagem individualmente. 
Para isso, a configuração de chama de propagação livre laminar em uma névoa de gotas foi simulada pelo grupo de pesquisa em \cite{SacomanoF2018CTM, SacomanoF2019IJHMT} no software CHEM1D \cite{Sommers1994PhD}.
Nesse caso, as equações de conservação de massa, espécie e entalpia são \cite{SacomanoF2018CTM,SacomanoF2021Fluids, }
\begin{equation}
    \frac{d \dot m}{d s} = \dot S_V^L
\end{equation}
\begin{equation}
    \frac{\partial(\dot m Y_i)}{\partial s} -
    \frac{\partial}{\partial s} \left(
        \frac{\lambda}{\mathrm{Le}_i c_p} \frac{\partial Y_i}{\partial s}
    \right) =
    \dot \omega_i + \delta_{ik}S_V^L
\end{equation}
\begin{equation}
    \frac{\partial(\dot m h)}{\partial s}
    -
    \frac{\partial}{\partial s} \left(\frac{\lambda}{c_p} \frac{\partial h}{\partial s} \right)
    =
    \frac{\partial}{\partial s} \left(
            \frac{\lambda}{c_p}\sum_{i=1}^{N_s}
            \left(\frac{1}{\mathrm{Le}_i}-1\right)
            h_i \frac{\partial Y_i}{\partial s} 
        \right)
        +
    S_h^L
\end{equation}
em que $\dot S_V^L$ e $S_h^L$ são os termos de acoplamento de fase de massa e entalpia.
\citeonline{SacomanoF2018CTM} resolvem a química com o método FGM para explorar as capacidades e limitações desse método.
\Idem{SacomanoF2019IJHMT} utilizam química detalhada para mostrar que é possivel representar a mistura gasosa com um subconjunto reduzido de espécies.  

% Modelagem CHEM1D. 
% Modelagem LES.
% Experiência com DTF.

\subsubsection{Modelagem Química} \label{sec:chem}

A modelagem química de maior fidelidade é chamada de química detalhada.
Essa abordagem utiliza um mecanismo químico com várias espécies e reações elementares, cada uma com uma taxa de reação modelada, por exemplo, com uma equação de Ahrrenius, para calcular as taxas de consumo ou produção das espécies principais e a formação de poluentes.
Os mecanismos podem ter dezenas de espécies e centenas de reações, o que torna esse método caro computacionalmente.

Uma alternativa para reduzir o custo computacional é o método FGM (\emph{Flamelet Generated Manifold}). 
Nesse método, a química detalhada é calculada previamente em vários cenários diferentes e uma biblioteca é construída, a qual conecta uma situação inicial, determinada por variáveis de controle, a uma situação final, pós combustão.
Ambos os métodos mencionados são oriundos da combustão gasosa, de forma que dois parâmetros de controle são necessários para descrever completamente o \emph{manifold} \cite{PetersN2000}.
Entretanto, em \citeonline{SacomanoF2018CTM} usam três variáveis de controle, efeitos são considerados não adiabáticos na tabulação.
São elas a fração de mistura $z$ e a váriável de progresso da reação $Y_{RPV}$, definida como $Y_{RPV}= Y_{\text{CO}_2} / {M_{\text{CO}_2}}+  Y_{\text H_2 \text O} / {2.5 M_{\text H_2 \text O}} + Y_{\text{CO}} / {1.5 M_{\text{CO}}}$, e a entalpia $h$.
As equações de transporte para as variáveis de controle tem a forma da equação abaixo, onde $\psi \in \lbrace z, Y_{RPV}, h\rbrace$.
\begin{equation}
    \frac{\partial \rho\psi}{\partial t} +
    \frac{\partial \rho u \psi}{\partial s} =
    \frac{\partial}{\partial s} \left(
        \Gamma_\psi \frac{\partial \psi}{\partial s}
        \right) +
    \dot\omega_\psi +
    S_h^L
\end{equation} 

Já nos trabalhos LES com espessamento artificial de chama (ATF), como \cite{SacomanoF2017PhD, SacomanoF2017CF, SacomanoF2020CF}, o tabelamento FGM utiliza apenas duas vairáveis de controle, $z$ e $Y_{RPV}$.
Além disso, a equação de transporte dessas variáveis é modificada para o espessamento da chama, tendo a forma dada pela equação abaixo, com $\psi \in \lbrace z, Y_{RPV}\rbrace$.
\begin{equation}
    \frac{\partial \bar \rho \widetilde \psi}{\partial t} + 
    \frac{\partial \bar \rho \widetilde \psi \overline u_j}{\partial x_j} =
    \frac{\partial }{\partial x_j} \left[ \left(
    FE \frac{\bar\mu}{Sc_\psi} + (1-\Omega)\frac{\mu_t}{Sc_{t,\psi}}
    \right) \frac{\partial \widetilde \psi}{\partial x_j}
    \right] +
    \frac{E}{F}\widetilde{\dot{\omega}_\psi} + 
    \overline S_{\psi,\nu'}^{\text{Eul}}
\end{equation}

% Mencionar química detalhada para CHEM1D.
% Mencionar FGM e FGM com ATF

\subsection{Modelos de Evaporação e Condensação (MEC)} \label{sec:MEC}

% Desenvolver e testar esses modelos analíticos é o foco deste trabalho.

Na escala uma gota, o problema pode ser dividido em duas regiões segregadas: a gota, líquida; e o gás ambiente circundante. 
Cada região é governada por um conjunto de equações.
Ambas regiões podem ser resolvidas numericamente (simuladas) ou modeladas com um resultado analítico.
Ambas regiões podem ser modeladas com diferentes graus de fidelidade.
Trabalhos que resolvem numericamente tanto o interior quanto o exterior da gota são capazes de simular muitos efeitos físicos a um elevado custo computacional (veja \source{Resolved evap small scale.}).
Por viabilizar o uso em simulações na escala da chama de spray, mais comuns são modelos que usam um modelo analítico em uma região e simulam a outra.
Geralmente, uma solução analítica é utilizada para a descrição espacial da fase gasosa e as propriedades da gota são integradas no tempo. 

No que tange a região líquida, do interior da gota, a hipótese mais simples é assumir uma distribuição homogênea de temperatura e espécie no interior da gota e negligenciar recirculação.
Isso elimina a necessidade de modelar o interior da gota.
Para a região gasosa, é 
Dessa forma, a fase gasosa pode ser resolvida analíticamente e a evolução da gota integrada no tempo.
Essa é a base para os modelos apresentados na Seção \ref{sec:RMM}.

Diferentes abordagem existem para considerar o interior da partícula.
Algumas são discutidas na Seção \ref{sec:int}.


\subsubsection{Modelos com Interior de Gota Homogêneo} \label{sec:RMM}


O primerio MEC foi desenvolvido por Fuchs \cite{Fuchs1959} na década de 1960. 
Esse modelo considera apenas a difusão de massa de e para a gota, representando assim evaporação e condensação, obtendo uma taxa de variação mássica linearmente dependente da diferença de fração de combustível na superfície da gota e no ambiente \cite{Glassman2008}.
% Por considerar apenas o transporte por difusão, a taxa de variação da  massica de da gota, desse modelo dependende linearmente da diferença de fração mássica da espécie na superfície e no ambiente.

Porém, o fluxo de massa provocado pelo fenômeno de evaporação torna relevante o transporte de massa por convecção, chamado de \emph{Stefan flow} nesse contexto. 
Assim, o modelo de Stefan-Fuchs considera os transportes por condução e convecção para/da gota e por isso é muito mais utilizado que o modelo de Fuchs.
O fluxo de massa nesse caso tem uma dependência logarítmica com a diferença de fração mássica do combustível \cite{Glassman2008, Turns2000}.
Esse modelo usa o chamado número de transporte de Spalding, introduzido em \cite{Law1978}, que pode ser derivado da equação de temperatura ou de espécie, originando $B_T$ e $B_M$.

Além das hipóteses já mencionadas de gota homogênea, esférica e regime quasi-estacionário, ambos modelos de Fuchs e de Stefan-Fuchs assumem também um ambiente quiescente e desconsideram a inércia térmica da gota.
A hipótese de ambiente quiescente pode ser relaxada para ambientes levemente convectivos utilizando correlações experimentais, como as de Froessling e a de Ranz-Marshall. 
O efeito do \emph{Stefan flow} nas correlações experimentais para o fluxo de calor e massa para é partícula é considerado no modelo de Abramzon Sirignano \cite{Sirignano1989} utilizado correções adivindas da teoria de filme (\emph{film theory}).
Eles também relaxam a hipótese de número de Lewis unitário e de inércia térmica desprezível, fornecendo uma metodologia para calcular a fase transiente de aquecimento da gota.

% A última hipótese, por sua vez, significa que a parte térmica do modelo não é capaz de prever o período de aquecimento da gota, antes da evaporação.
% Isso altera significativamente a chama em larga escala, já que, dependo do combustível e do tamanho da partícula, o período de aquecimendo da gota pode ser comparável ao tempo de vida \source{}.

% Por esse motivo, Ambrazon e Sirignano \cite{Sirignano1989} melhoraram o modelo corrigindo o número de transporte de temperatura de Spalding com o fluxo líquido de calor para a gota, permitindo simular o período de aquecimento da gota.
% Atualizaram também a correlação entre os números de transporte de Spalding para incluir números de Lewis não unitários.
% Também utilizam a teoria de filme (\emph{film theory}) para considerar os efeitos do \emph{Stefan flow} na camada limite da partícula, corrigindo o uso das expressões experimentais para um ambiente convectivo. 

Todos os modelos apresentados até agora são baseados na hipótese de equilíbrio termodinâmico na interface líquido-gás.
Por outro lado, dois anos antes da correção de Abramzon e Sirignano, Bellan e Harstad \cite{BellanJ1987} desenvolveram o modelo que inclui a condição de não equilíbrio termodinâmico na interface.

Miller, Harstad and Bellan \cite{MillerR1998} compararam os modelos de Ambrazon-Sirignano e Bellan-Harstad e combinaram-os em uma única representação matemática.
Por isso, é um dos modelos mais utilizados para a simulação turbulenta de sprays.  
\todo{A sua comparação mostrou ...}

Todos os modelos apresentados até agora assumem que a fase líquida e a fase gasosa podem ser tratadas como contínuas.
Desfazendo-se dessa hipótese, o modelo de Hertz-Knudsen-Langmuir \source{Langmuir} propõe uma formula para a taxa de variação da massa da gota baseada em cinética.

Sazhin \cite{Sazhin2006} comparou o modelo Stefan-Fuchs (chamado de clássico) com o modelo de Abramzon-Sirignano e com correlações experimentais desenvolvidas para hidrocarbonetos alcanos.
Ele obteve que o modelo de Stefan-Fuchs obtém as maiores taxas de evaporação, enquanto as correlações obtém as taxas mais conservadoras; o modelo de Abramzon-Sirignano obtendo valores intermediários, mais próximos das correlações experimentais que do modelo clássico.

Sacomano~et.~al \cite{SacomanoF2019IJHMT} comparou os modelos de Abramzon-Sirignano e Bellan-Harstad usando a formulação de \cite{MillerR1998} em uma simulação com química detalhada no CHEM1D \cite{Sommers1994PhD}.
\todo{Resultado para modelo de evap.}
Também comparou diferentes modelos de pressão de vapor de combustível na superfície da gota.
\todo{Mencionar necessidade de CC/Raul} \todo{e resultado da comparação}.
% 
O primerio MEC foi desenvolvido por \cite{Fuchs1959} na década de 1960. 
Esse modelo considera apenas a difusão de massa de e para a gota, representando assim evaporação e condensação.
Por considerar apenas o transporte por difusão, a taxa de variação da  massica de da gota, $\dot m_A$, desse modelo dependende linearmente da diferença de fração mássica da espécie $A$ na superfície e no ambiente.
\begin{equation}
    \dot m_A = 2 \pi r_s \rho D \Sh (Y_{A,\infty} - Y_{A,s}) \label{eq:maxwell}
\end{equation}
% % Essa equação pode ser reescrita utilizando o número de Sherwood.
% % \begin{equation}
% %     \dot m_{A} = A_p \frac{\Sh}{d_p} \rho D (Y_{A, \infty} - Y_{A, s})  \label{eq:maxwell_Sh}
% % \end{equation}
Porém, o fluxo de massa provocado pelo fenômeno de evaporação torna relevante o transporte de massa por convecção, chamado de \emph{Stefan flow} nesse contexto. 
Assim, o modelo de Stefan-Fuchs considera os transportes por condução e convecção para/da gota e por isso é muito mais utilizado que o modelo de Fuchs.
O fluxo de massa nesse caso tem uma dependência logarítmica com a diferença de fração mássica do combustível. \cite{Glassman2008, Turns2000}
\begin{align}
    \dot m_{A} &= 2 \pi r_s \Sh \rho D \ln{(1 + B_M)}              \label{eq:m_evap_BM} \\
    \dot m_A   &= 2 \pi r_s \Nu \frac{\lambda}{c_p} \ln{(1 + B_T)} \label{eq:m_evap_BT}
\end{align}
Esse modelo usa o chamado número de transporte de Spalding, introduzido por \source{Spalding}, que pode ser derivado da equação de temperatura ou de espécie, originando $B_T$ e $B_M$ (para igualdade, ver \cite{Glassman2008}).
\begin{align}
    B_M &= \frac{Y_{A,s} - Y_{A,\infty}}{1 - Y_{A,s}} \label{eq:B_M} \\
    B_T &= \frac{c_p (T_\infty - T_s)}{h_\evap}       \label{eq:B_T}
\end{align} 
Ambos modelos de Fuchs e de Stefan-Fuchs assumem gotas esféricas, num ambiente quiescente, em regime quasi-estacionário, considerando o interior da gota com temperatura e concentração uniforme e desconsiderando a inércia térmica da gota. 

Essa última hipótese significa que a parte térmica do modelo não é capaz de prever o período de aquecimento da gota, antes da evaporação.
Isso altera significativamente a chama em larga escala, já que o período de aquecimendo da gota é comparável ao tempo de combustão \source{source?} \question{burn time = tempo de combustão?}
Por esse motivo, Ambrazon e Sirignano \cite{Sirignano1989} melhoraram o modelo corrigindo o número de transporte de temperatura de Spalding
e atualizando a correlação entre os números de transporte de Spalding para incluir números de Lewis não unitários.
\begin{equation}
    B_T = \frac{c_p (T_\infty - T_s)}{h_\evap - Q_{\net}/\dot m_A} \label{eq:B_T_AS}
\end{equation}
\begin{equation}
    B_T = (1 + B_M)^\phi - 1,
    \qquad
    \phi=\frac{c_{p,A}}{c_{p,l}}\frac{1}{\Le}
    \label{eq:B_M_B_T}
\end{equation}
O fluxo de massa no modelo de Ambrazon e Sirignano é corrigido considerando a teoria de filme (\emph{film theory}) e efeitos do \emph{Stefan flow} na camada limite da partícula, corrigindo o uso das expressões experimentais para um ambiente convectivo.
Nesse modelo, a taxa de variação da massa pode ser obtida pela equação \eqref{eq:m_evap_AS_BM} e então a taxa líquida de transferência de calor para a gota pela equação \eqref{eq:B_T_AS}.
\begin{align}
\dot m_{A} &= 2 \pi r_s \Sh^* \rho D \ln{(1 + B_M)}              
,\qquad
\Sh^* = \Sh_0 \frac{B_M}{\ln (1 + B_M)} (1 + B_M)^{-0,7} \label{eq:m_evap_AS_BM}
\\
\dot m_A   &= 2 \pi r_s \Nu^* \frac{\lambda}{c_p} \ln{(1 + B_T)} 
,\qquad
\Nu^* = \Nu_0 \frac{B_T}{\ln (1 + B_T)} (1 + B_T)^{-0,7} \label{eq:m_evap_AS_BT}
\end{align}

O modelo Abramzon-Sirignano é baseado na hipótese de equilíbrio termodinâmico na interface líquido-gás.
Por outro lado, dois anos antes, Bellan e Harstad \source{BellanHarstadt} desenvolveram o modelo que inclui a condição de não equilíbrio termodinâmico na interface.
\todo{Ele obteve a seguinte formulação}
\begin{equation}
    \text{Taxa de variacao massica Bellan-Harstad}
\end{equation}


Sazhin em \cite{Sazhin2006} comparou o modelo Stefan-Fuchs (chamado de clássico) com o modelo de Abramzon-Sirignano e com correlações experimentais desenvolvidas para hidrocarbonetos alcanos.
Ele obteve que o modelo de Stefan-Fuchs obtém as maiores taxas de evaporação, enquanto as correlações obtém as taxas mais conservadoras; o modelo de Abramzon-Sirignano obtendo valores intermediários, mais próximos das correlações experimentais que do modelo clássico.

Miller \source{Miller1999} comparou os modelos de Ambrazon-Sirignano e Bellan-Harstad e combinou-os em uma única representação matemática.
\todo{A sua comparação mostrou ...}

Sacomano \cite{SacomanoF2019IJHMT} comparou os modelos de AS e BL usando a formulação de Miller1999 \todo{na situação ...}. 
Também comparou diferentes modelos de pressão de vapor de combustível na superfície da gota.
\todo{Ele encontrou ...}

Seguindo uma abordagem totalmente diferente, baseada em cinética ao invés de fenômenos de transporte, o modelo de Hertz-Knudsen-Langmuir \source{Langmuir} propõe uma formula para a taxa de variação da massa da gota.
\todo{Formulação de Lamguir-Knudsen}
\begin{equation}
    \text{Formulação de Lamguir-Knudsen}
\end{equation}

% Por outro lado, outros desenvolvimentos na decáda de 1980 e 1990 focaram em reduzir as hipóteses presentes no modelo de Stefan-Fuchs, principalmente no tange à dependência dos coeficientes de transporte na temperatura em em números de Lewis não unitários.
% Um desses trabalhos é Fachini \source{FachiniF1999}.

% Já Ulzama \source{Ulzama2006} considerou efeitos transitórios na modelagem da fase gasosa, criando um modelo misto transitório-quasi-estacionário.
% Porém, obtiveram resultados parecidos com o modelo clássico.

Modelo Chiu \emph{renormalization theory} \source{Chiu1999 e outros} para DNS e modelo integral de FanL \source{FanL2021}.



A modelagem de MEC \textbf{multicomponente} está intrinsicamente ligada ao tópico da difusão diferencial.
Essa modelagem vem sido desenvolvida pelo grupo de pesquisa nos últimos anos.

\citeonline{SacomanoF2021Fluids} estuda a influência da difusão diferencial gasosa na oxi-combustão de metano diluído em vapor d'água. 
Foram comparados três modelos diferentes para o fluxo de calor e de espécie.
O combustível nesse caso é gasoso, porém o vapor d'água pode se condensar em zonas frias ou evaporar em zonas quentes, constituindo uma fase dispersa.
Para esta fase foi utilizado o modelo de Abramzon-Sirignano \cite{Sirignano1989} na formulação de Miller \cite{MillerR1998}.

No ano seguinte, uma formulação rigorosa e robusta para a troca de calor e massa em gotas multicomponente foi derivada a partir das equações fundamentais em \cite{SacomanoF2022IJHMT}.
Essa formulação inclui efeitos de difusão diferencial de forma detalhada, ao custo de exigir um solver iterativo para resolver um conjunto de equações não linear para o MEC.
Incluiu também efeitos de mistura não ideal, utilizado modelos como \todo{modelos não ideal}.
O aspecto da difusão diferencial modelo baseou-se em trabalhos como \cite{ToniniS2015IJTS, ZhangL2012Fuel}. 

Esse modelo foi testado em \cite{SacomanoF2024CF} para a combustão de névoa quiescente de etanol anidro em atmosfera úmida, formando uma chama lisa e laminar, no CHEM1D \cite{Sommers1994PhD}.
A consideração dos efeitos inclusos nesse modelo se mostrou relevante até para o etanol anidro, já que o ar úmido pode condensar na gota de etanol.

Em \cite{SacomanoF2025CF}, o modelo completo, chamado "Full-DD" (DD - \emph{differential diffusion}), foi comparado com um modelo intermediário desenvolvido por Wang \cite{WangC2013CF}, chamado "Partial-DD" e com o modelo clássico de Stefan-Fuchs, para avaliar o compromisso fidelidade versus custo computacional.
\todo{Encontrou-se}.

O modelo completo foi extendido mais ainda por \cite{SantosA2024IJHMT} utilizando a equação de Maxwell-Stefan para a difusão, ao invés da difusão de Fick utilizada em todos os outros trabalhos mencionados até agora.
Esse modelo baseou-se por exemplo em \cite{ToniniS2015IJTS}.

Por uma outra perspectiva, foco exclusivo foi dado em \cite{SantosA2023IJHMT} para a equação da temperatura durante o processo de evaporação e condensação.
Nesse trabalho, os autores mostram que a equação de temperatura é independente do modelo utilizado para a transferência de massa, podendo estes serem modelados de desacoplada.
Cosntatou-se também que efeitos multicomponentes, enquanto mais complicados na transferência de massa por causa da difusão diferencial, na transferência de calor podem ser considerados diretamente por meio de calores específicos adequados.

Quanto ao aspecto \textbf{não ideal da mistura}, \citeonline{SacomanoF2022IJHMT} usou os métodos \todo{...}

\source{WangW2024} e \source{ZanuttoC2019}.




\subsubsection{Modelos para o Interor da Gota} \label{sec:int}


\subsection{Modelos de Combustão de Gota Isolada (MCGI)} \label{sec:MCGI}

% \subsubsection{Gotas Mono e Multi-Componente}

% Por outro lado, outros desenvolvimentos na decáda de 1980 e 1990 focaram em reduzir as hipóteses presentes no modelo de Stefan-Fuchs, principalmente no tange à dependência dos coeficientes de transporte na temperatura em em números de Lewis não unitários.
% Um desses trabalhos é Fachini \source{ FachiniF1999}.

% Já Ulzama \source{Ulzama2006} considerou efeitos transitórios na modelagem da fase gasosa, criando um modelo misto transitório-quasi-estacionário.
% Porém, obtiveram resultados parecidos com o modelo clássico.

\subsubsection{Modelos de Modo de Combustão de Gotas}