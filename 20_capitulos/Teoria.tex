\section{Fundamentação Teórica}

% 2. Fundamentação Teórica (meio q Trabalhos preliminares)
%    1. Transferencia de calor e massa em gotículas (experiencia do grupo)
%       1. Multicomponente
%       2. Interior de gota
%    2. Single Droplet Burning (minha exp)
%       1. Monocomponente (contexto de sprays e meu trabalho)
%       2. Multicomponente (Sekularac e ArabkhalajA2024)
%       3. Probabilidade de acontecer
%    3. Turbulent spray combustion
%       1. Modelagem mecflu
%          1. LES, eqs de transporte, ATF
%       2. Modelagem da química
%          1. FGM



\subsection{Transferência de Calor e Massa em Gotas}

\subsubsection{Modelos de Evaporação e Condensação Monocomponente}


O primerio MEC foi desenvolvido por \cite{Fuks1959} na década de 1960. 
Esse modelo considera apenas a difusão de massa de e para a gota, representando assim evaporação e condensação.
Por considerar apenas o transporte por difusão, a taxa de variação da  massica de da gota, desse modelo dependende linearmente da diferença de fração mássica da espécie na superfície e no ambiente.

Porém, o fluxo de massa provocado pelo fenômeno de evaporação torna relevante o transporte de massa por convecção, chamado de \emph{Stefan flow} nesse contexto. 
Assim, o modelo de Stefan-Fuchs considera os transportes por condução e convecção para/da gota e por isso é muito mais utilizado que o modelo de Fuchs.
O fluxo de massa nesse caso tem uma dependência logarítmica com a diferença de fração mássica do combustível. \source{Williams, Turns}
Esse modelo usa o chamado número de transporte de Spalding, introduzido por \source{Spalding}, que pode ser derivado da equação de temperatura ou de espécie, originando $B_T$ e $B_M$.

Ambos modelos de Fuchs e de Stefan-Fuchs assumem gotas esféricas, num ambiente quiescente, em regime quasi-estacionário, considerando o interior da gota com temperatura e concentração uniforme e desconsiderando a inércia térmica da gota.
A hipótese quiescente pode ser relaxada para ambientes levemente convectivos utilizando correlações experimentais, como as de Froessling e a de Ranz-Marshall. 
A última hipótese, por sua vez, significa que a parte térmica do modelo não é capaz de prever o período de aquecimento da gota, antes da evaporação.
Isso altera significativamente a chama em larga escala, já que, dependo do combustível e do tamanho da partícula, o período de aquecimendo da gota pode ser comparável ao tempo de combustão \source{source?} \question{burn time = tempo de combustão?}

Por esse motivo, Ambrazon e Sirignano \source{Sirignano1989} melhoraram o modelo corrigindo o número de transporte de temperatura de Spalding com o fluxo líquido de calor para a gota, permitindo simular o período de aquecimento da gota.
Atualizaram também a correlação entre os números de transporte de Spalding para incluir números de Lewis não unitários.
Também utilizam a teoria de filme (\emph{film theory}) para considerar os efeitos do \emph{Stefan flow} na camada limite da partícula, corrigindo o uso das expressões experimentais para um ambiente convectivo. 

Todos os modelos apresentados até agora são baseados na hipótese de equilíbrio termodinâmico na interface líquido-gás.
Por outro lado, dois anos antes, Bellan e Harstad \source{BellanHarstadt} desenvolveram o modelo que inclui a condição de não equilíbrio termodinâmico na interface.

Miller \source{Miller1999} comparou os modelos de Ambrazon-Sirignano e Bellan-Harstad e combinou-os em uma única representação matemática.
Por isso, é um dos modelos mais utilizados para a simulação turbulenta de sprays.  
\todo{A sua comparação mostrou ...}

Todos os modelos apresentados até agora assumem que a fase líquida e a fase gasosa podem ser tratadas como contínuas.
Desfazendo-se dessa hipótese, o modelo de Hertz-Knudsen-Langmuir \source{Langmuir} propõe uma formula para a taxa de variação da massa da gota baseada em cinética.

% Sazhin em \source{Sazhin2006PECS} comparou o modelo Stefan-Fuchs (chamado de clássico) com o modelo de Abramzon-Sirignano e com correlações experimentais desenvolvidas para hidrocarbonetos alcanos.
% Ele obteve que o modelo de Stefan-Fuchs obtém as maiores taxas de evaporação, enquanto as correlações obtém as taxas mais conservadoras; o modelo de Abramzon-Sirignano obtendo valores intermediários, mais próximos das correlações experimentais que do modelo clássico.


Sacomano \source{Sacomano2019} comparou os modelos de AS e Miller usando a formulação de Miller1999 \todo{na situação ...}. 
Também comparou diferentes modelos de pressão de vapor de combustível na superfície da gota.
\todo{Ele encontrou ...}


% Modelo Chiu \emph{renormalization theory} \source{Chiu1999 e outros} para DNS e modelo integral de FanL \source{FanL2021}.

% 
O primerio MEC foi desenvolvido por \cite{Fuchs1959} na década de 1960. 
Esse modelo considera apenas a difusão de massa de e para a gota, representando assim evaporação e condensação.
Por considerar apenas o transporte por difusão, a taxa de variação da  massica de da gota, $\dot m_A$, desse modelo dependende linearmente da diferença de fração mássica da espécie $A$ na superfície e no ambiente.
\begin{equation}
    \dot m_A = 2 \pi r_s \rho D \Sh (Y_{A,\infty} - Y_{A,s}) \label{eq:maxwell}
\end{equation}
% % Essa equação pode ser reescrita utilizando o número de Sherwood.
% % \begin{equation}
% %     \dot m_{A} = A_p \frac{\Sh}{d_p} \rho D (Y_{A, \infty} - Y_{A, s})  \label{eq:maxwell_Sh}
% % \end{equation}
Porém, o fluxo de massa provocado pelo fenômeno de evaporação torna relevante o transporte de massa por convecção, chamado de \emph{Stefan flow} nesse contexto. 
Assim, o modelo de Stefan-Fuchs considera os transportes por condução e convecção para/da gota e por isso é muito mais utilizado que o modelo de Fuchs.
O fluxo de massa nesse caso tem uma dependência logarítmica com a diferença de fração mássica do combustível. \cite{Glassman2008, Turns2000}
\begin{align}
    \dot m_{A} &= 2 \pi r_s \Sh \rho D \ln{(1 + B_M)}              \label{eq:m_evap_BM} \\
    \dot m_A   &= 2 \pi r_s \Nu \frac{\lambda}{c_p} \ln{(1 + B_T)} \label{eq:m_evap_BT}
\end{align}
Esse modelo usa o chamado número de transporte de Spalding, introduzido por \source{Spalding}, que pode ser derivado da equação de temperatura ou de espécie, originando $B_T$ e $B_M$ (para igualdade, ver \cite{Glassman2008}).
\begin{align}
    B_M &= \frac{Y_{A,s} - Y_{A,\infty}}{1 - Y_{A,s}} \label{eq:B_M} \\
    B_T &= \frac{c_p (T_\infty - T_s)}{h_\evap}       \label{eq:B_T}
\end{align} 
Ambos modelos de Fuchs e de Stefan-Fuchs assumem gotas esféricas, num ambiente quiescente, em regime quasi-estacionário, considerando o interior da gota com temperatura e concentração uniforme e desconsiderando a inércia térmica da gota. 

Essa última hipótese significa que a parte térmica do modelo não é capaz de prever o período de aquecimento da gota, antes da evaporação.
Isso altera significativamente a chama em larga escala, já que o período de aquecimendo da gota é comparável ao tempo de combustão \source{source?} \question{burn time = tempo de combustão?}
Por esse motivo, Ambrazon e Sirignano \cite{Sirignano1989} melhoraram o modelo corrigindo o número de transporte de temperatura de Spalding
e atualizando a correlação entre os números de transporte de Spalding para incluir números de Lewis não unitários.
\begin{equation}
    B_T = \frac{c_p (T_\infty - T_s)}{h_\evap - Q_{\net}/\dot m_A} \label{eq:B_T_AS}
\end{equation}
\begin{equation}
    B_T = (1 + B_M)^\phi - 1,
    \qquad
    \phi=\frac{c_{p,A}}{c_{p,l}}\frac{1}{\Le}
    \label{eq:B_M_B_T}
\end{equation}
O fluxo de massa no modelo de Ambrazon e Sirignano é corrigido considerando a teoria de filme (\emph{film theory}) e efeitos do \emph{Stefan flow} na camada limite da partícula, corrigindo o uso das expressões experimentais para um ambiente convectivo.
Nesse modelo, a taxa de variação da massa pode ser obtida pela equação \eqref{eq:m_evap_AS_BM} e então a taxa líquida de transferência de calor para a gota pela equação \eqref{eq:B_T_AS}.
\begin{align}
\dot m_{A} &= 2 \pi r_s \Sh^* \rho D \ln{(1 + B_M)}              
,\qquad
\Sh^* = \Sh^0 \frac{B_M}{\ln (1 + B_M)} (1 + B_M)^{-0,7} \label{eq:m_evap_AS_BM}
\\
\dot m_A   &= 2 \pi r_s \Nu^* \frac{\lambda}{c_p} \ln{(1 + B_T)} 
,\qquad
\Nu^* = \Nu^0 \frac{B_T}{\ln (1 + B_T)} (1 + B_T)^{-0,7} \label{eq:m_evap_AS_BT}
\end{align}

O modelo Abramzon-Sirignano é baseado na hipótese de equilíbrio termodinâmico na interface líquido-gás.
Por outro lado, dois anos antes, Bellan e Harstad \source{BellanHarstadt} desenvolveram o modelo que inclui a condição de não equilíbrio termodinâmico na interface.
\todo{Ele obteve a seguinte formulação}
\begin{equation}
    \text{Taxa de variacao massica Bellan-Harstad}
\end{equation}


Sazhin em \cite{Sazhin2006} comparou o modelo Stefan-Fuchs (chamado de clássico) com o modelo de Abramzon-Sirignano e com correlações experimentais desenvolvidas para hidrocarbonetos alcanos.
Ele obteve que o modelo de Stefan-Fuchs obtém as maiores taxas de evaporação, enquanto as correlações obtém as taxas mais conservadoras; o modelo de Abramzon-Sirignano obtendo valores intermediários, mais próximos das correlações experimentais que do modelo clássico.

Miller \source{Miller1999} comparou os modelos de Ambrazon-Sirignano e Bellan-Harstad e combinou-os em uma única representação matemática.
\todo{A sua comparação mostrou ...}

Sacomano \cite{SacomanoF2019IJHMT} comparou os modelos de AS e BL usando a formulação de Miller1999 \todo{na situação ...}. 
Também comparou diferentes modelos de pressão de vapor de combustível na superfície da gota.
\todo{Ele encontrou ...}

Seguindo uma abordagem totalmente diferente, baseada em cinética ao invés de fenômenos de transporte, o modelo de Hertz-Knudsen-Langmuir \source{Langmuir} propõe uma formula para a taxa de variação da massa da gota.
\todo{Formulação de Lamguir-Knudsen}
\begin{equation}
    \text{Formulação de Lamguir-Knudsen}
\end{equation}

% Por outro lado, outros desenvolvimentos na decáda de 1980 e 1990 focaram em reduzir as hipóteses presentes no modelo de Stefan-Fuchs, principalmente no tange à dependência dos coeficientes de transporte na temperatura em em números de Lewis não unitários.
% Um desses trabalhos é Fachini \source{FachiniF1999}.

% Já Ulzama \source{Ulzama2006} considerou efeitos transitórios na modelagem da fase gasosa, criando um modelo misto transitório-quasi-estacionário.
% Porém, obtiveram resultados parecidos com o modelo clássico.

Modelo Chiu \emph{renormalization theory} \source{Chiu1999 e outros} para DNS e modelo integral de FanL \source{FanL2021}.


\subsubsection{Modelos de Evaporação e Condensação Multicomponente}

\subsubsection{Modelos para o Interor da Gota}




\subsection{Combustão de Gota Isolada}

\subsubsection{Gotas Mono- e Multi-Componente}

% Por outro lado, outros desenvolvimentos na decáda de 1980 e 1990 focaram em reduzir as hipóteses presentes no modelo de Stefan-Fuchs, principalmente no tange à dependência dos coeficientes de transporte na temperatura em em números de Lewis não unitários.
% Um desses trabalhos é Fachini \source{ FachiniF1999}.

% Já Ulzama \source{Ulzama2006} considerou efeitos transitórios na modelagem da fase gasosa, criando um modelo misto transitório-quasi-estacionário.
% Porém, obtiveram resultados parecidos com o modelo clássico.


\subsubsection{Modelos de Modo de Combustão de Gotas}




\subsection{Combustão Turbulenta de Sprays}

\subsubsection{Modelagem da Fase Contínua}

Modelagem CHEM1D

Modelagem LES

\subsubsection{Modelagem Química}

Mencionar química detalhada para CHEM1D

Introdução FGM e ATF