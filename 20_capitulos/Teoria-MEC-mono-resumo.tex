
O primerio MEC foi desenvolvido por \source{Fuchs} na década de 1960. 
Esse modelo considera apenas a difusão de massa de e para a gota, representando assim evaporação e condensação.
Por considerar apenas o transporte por difusão, a taxa de variação da  massica de da gota, desse modelo dependende linearmente da diferença de fração mássica da espécie na superfície e no ambiente.

Porém, o fluxo de massa provocado pelo fenômeno de evaporação torna relevante o transporte de massa por convecção, chamado de \emph{Stefan flow} nesse contexto. 
Assim, o modelo de Stefan-Fuchs considera os transportes por condução e convecção para/da gota e por isso é muito mais utilizado que o modelo de Fuchs.
O fluxo de massa nesse caso tem uma dependência logarítmica com a diferença de fração mássica do combustível. \source{Williams, Turns}
Esse modelo usa o chamado número de transporte de Spalding, introduzido por \source{Spalding}, que pode ser derivado da equação de temperatura ou de espécie, originando $B_T$ e $B_M$.

Ambos modelos de Fuchs e de Stefan-Fuchs assumem gotas esféricas, num ambiente quiescente, em regime quasi-estacionário, considerando o interior da gota com temperatura e concentração uniforme e desconsiderando a inércia térmica da gota. 
Essa última hipótese significa que a parte térmica do modelo não é capaz de prever o período de aquecimento da gota, antes da evaporação.
Isso altera significativamente a chama em larga escala, já que o período de aquecimendo da gota é comparável ao tempo de combustão \source{source?} \question{burn time = tempo de combustão?}
Por esse motivo, Ambrazon e Sirignano \source{Sirignano1989} melhoraram o modelo corrigindo o número de transporte de temperatura de Spalding
e atualizando a correlação entre os números de transporte de Spalding para incluir números de Lewis não unitários.

Além disso, os autores utilizam-se da a teoria de filme (\emph{film theory}) para considerar os efeitos do \emph{Stefan flow} na camada limite da partícula, corrigindo o uso das expressões experimentais para um ambiente convectivo. 

O modelo Abramzon-Sirignano é baseado na hipótese de equilíbrio termodinâmico na interface líquido-gás.
Por outro lado, dois anos antes, Bellan e Harstad \source{BellanHarstadt} desenvolveram o modelo que inclui a condição de não equilíbrio termodinâmico na interface.

Sazhin em \source{Sazhin2006PECS} comparou o modelo Stefan-Fuchs (chamado de clássico) com o modelo de Abramzon-Sirignano e com correlações experimentais desenvolvidas para hidrocarbonetos alcanos.
Ele obteve que o modelo de Stefan-Fuchs obtém as maiores taxas de evaporação, enquanto as correlações obtém as taxas mais conservadoras; o modelo de Abramzon-Sirignano obtendo valores intermediários, mais próximos das correlações experimentais que do modelo clássico.

Miller \source{Miller1999} comparou os modelos de Ambrazon-Sirignano e Bellan-Harstad e combinou-os em uma única representação matemática.
\todo{A sua comparação mostrou ...}

Sacomano \source{Sacomano2019} comparou os modelos de AS e BL usando a formulação de Miller1999 \todo{na situação ...}. 
Também comparou diferentes modelos de pressão de vapor de combustível na superfície da gota.
\todo{Ele encontrou ...}

Seguindo uma abordagem totalmente diferente, baseada em cinética ao invés de fenômenos de transporte, o modelo de Hertz-Knudsen-Langmuir \source{Langmuir} propõe uma formula para a taxa de variação da massa da gota.

% Modelo Chiu \emph{renormalization theory} \source{Chiu1999 e outros} para DNS e modelo integral de FanL \source{FanL2021}.
