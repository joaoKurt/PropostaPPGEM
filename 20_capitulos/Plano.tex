\section{Plano de Trabalho e Cronograma de Excecução}


Assim, podem ser determinadas as seguintes etapas do projeto:

\begin{enumerate}
    \item Modelagem  analítica de combustão de gota isolada monocomponente com modelo avançado de evaporação;
    \item Modelagem  analítica de discretização do interior de gota monocomponente;
    \item Acoplamento de modelos monocomponente de combustão de gota isolada com discretização no interior da gota;
    \item Modelagem analítica de combustão de gota isolada multicomponente com modelo avançado de evaporação;
    \begin{description}
        \item[Nota:] Deve-se diferenciar aqui diferentes possibilidades: como gota bicomponente sendo apenas um o combustível, exemplo etanol anidro, e gota bi- ou multicomponente com mais de um componente volátil e combustível.
        \item[Adendo:] Essa etapa pode eventualmente ser dividida em mais de uma etapa, devido aos diferentes cenários possíveis. 
    \end{description}
    \item Modelagem analítica de discretização do interior de gota multicomponente;
    \item Acoplamento de modelos multicomponente de combustão de gota isolada com discretização no interior da gota;
    \item Estudar modelos de determinação de probabilidade de gotas entrarem no modo de combustão isolada;
    \item Implementar novos modelos no CHEM1D;
    \item Implementar novos modelos no OpenFOAM;
    \item Estudar como novo modelo se encaixa no contexto de interação gota-chama;
    \item Estudo de chamas turbulentas com todos novos modelos acoplados.
\end{enumerate}

Tabela com cronograma.

\subsection{Disciplinas a serem cursadas}

Necessário?