\section{Metodologia}

%% Materiais e Métodos - Software e Hardware

No contexto de desenvolvimento de modelos analíticos de HMT, o projeto visa  incluir os seguintes aspectos:
(i) modelo de combustão de gota isolada;
(ii) aspecto multicomponente; 
(iii) modelagem do interior da gota. 
Para atingir esse objetivo, o desenvolvimento será gradual e dividido em etapas. 
Cada modelo analítico será desenvolvido sozinho, em seguida integrado com as outras capacidades.
Para cada um dos três aspectos listados acima, serão realizadas as seguintes etapas:
\begin{enumerate}
    \item[A.] Busca e análise de modelos já existentes na listeratura;
    \item[B.] Desenvolvimento analítico do novo modelo;
    \item[C.] Implementação do novo modelo no CHEM1D;
    \item[D.] Simulação e análise dos resultados, inclindo avaliação de desempenho do modelo.
\end{enumerate}
A cada nova capacidade adicionada ao modelo, as anteriores serão mantidas, de modo que este se torna cada vez mais complexo e abrangente. 
Após desenvolver, implementar e testar os modelos no CHEM1D, prevê-se a implementação do modelo no OpenFOAM. Desse modo, faz-se necessárias as seguintes etapas:
\begin{itemize}
    \item[E.] Estudar como implementar modelos no CHEM1D; 
    \item[F.] Estudar C++; 
    \item[G.] Estudar como implementar modelos no OpenFOAM; 
\end{itemize}

